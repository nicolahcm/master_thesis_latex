
Delle correzioni di GC per chap 4 ho finito, apparte il test da eseguire finale (al di fuori del dominio di convergenza).
Rimane:

	0) Sistemare le immagini solo , visto che il testo sembra giusto. ---> TROVATO SOL PER IMMAGINI: metti "ht" in begin figure!!!
	1) Rileggere il testo e sistemare le virgole (ne mancano parecchie).
	2) (described in Subsection 3.3.1 and equation (3.13)) is useless  <---- Fix here "useless".
	
	

	2) Aggiungere il test addizionale suggerito dalla Callegaro, l'espansione dopo T, anche se non ci serve perché la sequenza
	è solo per approssimare psi valutato in T. 
	Mettilo come sottosezione e aggiungi il grafico più 2 paroline.
	----> NOO: dire che nei grafici di psi n quello non approssima psi in tutti quei punti ma solo in psi(T).
	
	

&&& Terminologia migliore... 
	## Obbligatorio da fare:
	
	## Facoltativo in futuro o già completati.
	
	6) Controlla dove uso "remind" e sostituisci con "recall"
	5) Controlla dove ho messo la frawe "assume true" e sostituisci con "assume"
	7) Usa "presented" quando dici questi dati sono presentati in questo plot o tavola. 

Facoltativo:

1) Some comments on the time of execution with the simplified version (dividing both sides by $\lambda$)... Forse posso farlo.
2) To add the fact that knowing the characteristic function, we can get the real function through Fourier techniques.
3) Aggiungere nei test numerici: cosa succede se alpha è tale che non c'è il comportamento "rough"? 

