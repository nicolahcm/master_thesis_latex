


1) To add the fact that knowing the characteristic function, we can get the real function through Fourier techniques. 

2) dire nel RR extrapolation:
2.1) triviale che entrambi convergono a psi
2.2) L'unica differenza tra i 2 rr extrapolation sono le velocità di convergenza.
2.3) accennare che nei test numerici sembrerà che il suddetto rr3 rate convergence rate sembra non valere, ma il rr2 si.
2.4) Anche se il convergence rate di prima sembra non valere, si vede che però rr3 converge sempre prima di rr2.

3) Sistemare la prima parte di RR extrapolt. nei test numerici.
4) Collegarla alla parte di RR nei test numerici.

5) RR3 lo chiamo double step RR extrapolation solo nei tesi numerici. Dovrei chiamarlo così anche prima.
